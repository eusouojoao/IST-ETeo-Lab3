%\clearpage
%//==============================--@--==============================//%
\subsection*{\underline{3.1} Verifique que o factor de qualidade $Q_0$ num circuito RLC-série é a sobretensão de $U_{Lef}$ ou $U_{Cef}$ em relação à tensão aplicada $U_{Gef}$ na situação de ressonância.}
\label{subsection3_1}
\paragraph{Resposta:}
O factor de qualidade $Q_0$  é uma característica de um circuito ressonante (como é o caso do circuito RLC-série) que mede a qualidade dos componentes reactivos. Este factor indica-nos a relação entre a energia armazenada e a energia dissipada:

$$ Q_0 = 2\pi \cdot \frac{\text{(Energia máxima armazenada)}}{\text{(Energia perdida p/ ciclo)}} $$

Note-se que as perdas de energia por ciclo nada mais são do que a potência média dissipada por período, i.e., temos:

$$ Q_0 = 2\pi \cdot \frac{\text{(Energia máxima armazenada)}}{\text{(Potência média dissipada)}\cdot T_0} = \omega_0 \cdot \frac{\text{(Energia máxima armazenada)}}{\text{(Potência média dissipada)}}$$

Do Teorema de Poynting complexo é-nos trivial deduzir que no estado de ressonância, a potência reactiva é \underline{nula}. Isto indica-nos que tanto a bobina como o condensador estão a produzir a mesma energia, mas como estão em oposição de fase, anulam-se. Sendo assim, olhamos para o circuito como puramente resistivo. (Observa-se que $\vert \bar{Z}_{eq}\vert = R \land \vert \bar{I}\vert = \vert\bar{I}\vert_{MAX} = \frac{U_{G}}{R}$).

$$ 
\implies Q_0 = \omega_o \frac{(W_m)_{MAX}}{(p_J)_{av}} = \omega_0 \frac{\frac{1}{2}LI_{ef}^2}{R\ (\frac{I_{ef}}{\sqrt{2}})^2} = \frac{\omega_0 L}{R}\\ 
$$

$$
\implies Q_0 = \omega_0 \frac{(W_e)_{MAX}}{(p_J)_{av}} = \omega_0 \frac{\frac{1}{2}C U_{ef}^2}{R\ (\frac{I_{ef}}{\sqrt{2}})^2} = \omega_0 \frac{\frac{1}{2} C I_{ef}^2 \mathcal{X}_c^2}{\frac{1}{2}I_{ef}^2\ R} = \frac{1}{\omega_0 RC}\ \text{, com}\ \mathcal{X}_c = \frac{1}{\omega_0 C}
$$

$$
\therefore Q_0 = \frac{\omega_0 L}{R} = \frac{1}{\omega_0 RC} = \frac{1}{R}\sqrt{\frac{L}{C}}
$$
\hfill \ensuremath{\Box}

No estado de ressonância as tensões no condensador e na bobina são de igual amplitude; verifica-se igualdade (em módulo) das reactâncias destes componentes.

$$
\begin{dcases*}
U_L = I \mathcal{X}_L = I\cdot \omega_0 L = \frac{\omega_0 L}{R}\cdot U_G = Q_0\cdot U_G\\
U_C = I \mathcal{X}_C = \frac{I}{\omega_0 C} = \frac{1}{\omega_0 RC}\cdot U_G = Q_0\cdot U_G
\end{dcases*}
$$

Justaposto, trivialmente concluímos que em ressonância: 
$$ 
Q_0 = \dfrac{U_L}{U_G} = \dfrac{U_C}{U_G} \iff Q_0 = \dfrac{U_{Lef}}{U_{Gef}} = \dfrac{U_{Cef}}{U_{Gef}}
$$
\hfill \ensuremath{\Box}
%//==============================--@--==============================//%
